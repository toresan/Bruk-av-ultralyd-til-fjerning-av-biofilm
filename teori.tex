\section{Teori}
\subsection{Biofilm}
Ein biofilm er bakteriar som har ``grodd fast'' på ei eller anna overflate og har danna eit mikrobiologisk økosystem som er meir motstandsdyktig enn vanlege bakteriar åleine og biofilm kan bli danna på ei rekkje ulike overflater \cite{biofilm}\cite{biofilm2}. Dette vesle økosystemet utviklar seg til å bli noko meir enn bakteriar og innheld ofte alger og protozo. Figur \ref{fig:biofilm} viser korleis utviklinga av biofilm går føre seg og det er særleg når biofilmen når det fjerde stadiet at den kan bli skadeleg. Økosystemet kan vere med på å lage nye bakteriar som t.d. Legionella \cite{biofilm} og dersom biofilmen oppstår i kroppen vår kan dette skape infeksjonar.

\begin{figure}[htbp]
  \includegraphics{bilete/DanningAvBiofilm.pdf}
  \caption[Dei fem fasene i biofilmdanning]{1) Bakterien finn ein overflate. 2) Bakterien festar seg permanent. 3) Bakterien er i første fase av å danne biofilm 4) Biofilmen blir danna ferdig og skapar nye bakteriar. 5) Biofilmen spreier nye bakteriar ut til omgivnadane. Figuren er henta frå \cite{biofilmfigur}.}
  \label{fig:biofilm}
\end{figure}

Særleg i samband med implantering av proteser, som t.d. hoftekuler eller kne, har det vist seg at biofilm er eit problem \cite{ultraprotese}. Under operasjon anten ved innsetting eller ved vedlikehald av protesene kan det lett kome bakteriar på protesen som etterkvart dannar ein biofilm på protesen som er særs motstandsdyktig mot immunsystemet. Denne biofilmen blir så årsaka til infeksjonar som kan oppstå fleire månedar etter operasjonen. 

For å gjere pasienten frisk frå ein slik infeksjon er det som regel ikkje nok å berre gi pasienten medisin, det må ein ny operasjon til for å fjerne infeksjonen og biofilmen\cite{infection}. Per i dag er det ingen billig god måte å få diagnose på ein slik infeksjon, men det byrjar å kome fleire forsøk som viser at ein kan nytte ultralyd i sambinding med diagnostiseringa og kanskje også til å fjerne biofilm fullstendig frå proteser \cite{ultraprotese}.

\subsection{Ultralyd}
\subsubsection*{Generelt}
Ultralyd er i \cite[s. 443]{citeulike:4412590} definert som all lyd med frekvens over \unit{20}{\kilo \hertz} og i dette prosjektet skal vi særleg halde oss i frekvensområdet \unit{20 - 200}{\kilo \hertz}. Og sidan alle forsøka i dette prosjektet går føre seg under vatn, vil bølgjelengda gitt ved $\lambda = c / f$, der $c = \unit{1481}{\meter / \second}$, bli mellom \unit{7 - 74}{\milli\meter}.

\subsubsection*{Medisinsk bruk i dag}
Vi har i dag ein utstrakt bruk av ultralyd innan medisin og det fins mange bruksområder for ultralyd. Mellom anna blir ultralyd nytta til medisinsk avbilding i samband med diagnostikk, som t.d. undersøking av gravide eller pasientar med hjertefeil, og ultralyden blir då nytta som eit slags ekkolodd for å sjå inn i kroppen vår. Ultralyd er også vanleg i meir terapautiske samanhengar. Nyresteinar kan fjernast ved å knuse dei med ultralyd som har høg intensitet og som er fokusert spesielt mot nyresteinen. Elles kan ein også nemne at ultralyd er blir nytta av tannlækjarar til å reinse tenner.

\subsection{Undervassakustikk}

\subsubsection*{Generelt}

\subsubsection*{Lydfarten}

\subsection{Kavitasjon}
Kavitasjon oppstår når trykket på lydbølgja overgår det hydrostatiske trykket i vatnet.\cite{Kinsler:2000rc}

Kavitasjonsterskelen er grovt gitt\cite{kavitasjon} med formel \eqref{kavitasjonsterskel} eller \eqref{kavitasjonsterskel2}

\begin{equation}
\label{kavitasjonsterskel}
P_t = 0.06 + 0.05f
\end{equation}

\begin{eqnarray}
	\label{kavitasjonsterskel2}
	\frac{p^2_{rms}}{\rho c} = I \\
	p_{rms} = \sqrt{\rho* c * i} \\
	p = \sqrt{998* 1481 * i * 2} 
\end{eqnarray}


XXX KAVITASJONSTERSKEL

XXX FIGUR AV KAVITASJONSTERSKEL

XXX BILETE AV KAVITASJON?

\subsection{Tidlegare arbeid FLYTTE?}
XXX UTSTYR?

XXX KORLEIS VIRKAR DET ENKLE UTSTYRET? FREKVENSSPEKTRUM

XXX KVA PARAMETRAR SÅG DEI PÅ

XXX MANGELFULLT, KVA SKULLE EIN YNSKT Å HA SETT PÅ?
