\section{Diskusjon}
\subsection{Analyse av resultat}
Sidan dette prosjektet tek føre seg eit mogleg måleoppsett til bruk i vidare arbeid om ultralyd sin effekt på biofilm er det vanskeleg å seie så mykje om resultata som er oppnådd i dette prosjektet. Likevel kan vi seie noko om det ein har funne ut så langt. I apparatet som har blitt nytta tidlegare i forskinga rundt ultralyd og biofilm, så er det tydeleg at dette apparatet sin styrke ligg i å kunne sende ut lydbølgjer kontinuerleg over lengre tid. Oppsettet eg har laga har ei avgrensing i høve til transduceren eg har nytta, då den ikkje toler å stå på kontinuerleg med tilsvarande effekt. Denne skilnaden er ikkje ynskjeleg i et måleoppsett sidan apparatet har hatt noko suksess med å fjerne/drepe biofilmbakteriar. Det er viktig å undersøkje om det er dei kontinuerlege lydbølgjene som gir den ynskja effekten på biofilmen. I tillegg kunne ein observere kavitasjon i apparatet medan i mitt måleoppsett blei dette aldri observert. Dette er svært uheldig då det er kavitasjon som er den viktigaste effekten i andre ultralydsamanhengar der ein nyttar den til å reinse noko. Kvifor kaviterte då aldri vatnet med mitt eige måleoppsett? Det skal ikkje vere nødvendig å la transduceren stå på med full effekt i fleire minutt for å få kavitasjon. Då er det nok heller intensiteten det er noko i veien med. Anten burde eg ha ein transducer som treng mindre effekt per eining ut eller så må eg ha ein effektforsterker som har høgare forsterking for å kunne få kavitasjon.

Det positive med måleoppsettet mitt er at eg har full kontroll med frekvensen og lydtrykket\footnote{I alle fall opp til eit visst nivå har eg kontroll over lydtrykket. Forsterkaren har maksimal inngangsspenning på \unit{1,4}{\volt} Pk-Pk, og ein kan dermed variere både frekvens og lydtrykk, men eit endå større dynamisk område er ynskjeleg.} noko som manglar på apparatet brukt i \cite{ultraprotese}. Tidlegare har det ikkje vore gjort målingar av lydtrykk og frekvens i det heile, så berre det å ha ein hydrofon i dei omgivnadane ein protese vil bli plassert i er ei forbedring frå tidlegare målemetode.
\subsection{Analyse av oppsett}
Eit problem med oppsettet mitt kom fram i den store variasjonen i lydtrykket som funksjon av hydrofonplasseringa som blei observert på oscilloskopet. Berre ved å flytte hydrofonen noke millimeter kunne lydtrykket endre seg drastisk. Dette i seg sjølv har ikkje så stor betyding dersom eg berre skulle ha målt utan glaskolbe til dømes, men i og med at eg er nøydd til å kunne måle både med og utan så blir dette lett eit problem. Sjølv om eg nyttar stativ til å plassere hydrofonen, så er det særs lett at det blir avvik i posisjon når eg må bytte til å glaskolbe rundt hydrofonen. Her kunne nok eit bedre stativ vore løysinga.

Ser ein på Sonorexapparatet så ser ein at det har to transducerar i motsetnad til min eine transducer. Det får meg til å tenkje at det hadde vore interessant å prøve mange transducerar i ein tank for å sjå effekten av eit jamnare lydfelt i tanken. Det hadde også vore ynskjeleg med fleire hydrofonar for å kunne måle på ulike posisjonar samstundes utan å måtte flytte på utstyret for kvar måling.

\subsection{Problem undervegs}
Det var diverse problem i løpet av prosjektets gang som har seinka arbeid og gjort det umogleg å teste vidare tankar om måleoppsettet. Mellom anna, så måtte den tilgjengelege effektforsterkaren som var kraftigare enn den eg har nytta, bli sendt på reparasjon og denne reparasjonen tok så lang tid at i det den kom attende var det for seint å byrje nye målingar med den.  

I tillegg hadde eg lenge problem med målingar via PC. For alle målingane eg hadde gjort såg det ut til at lydtrykksnivået aldri oversteig \unit{170}{\deci\bel} re \unit{1}{\micro\pascal}\footnote{Dette er berre \unit{10}{\deci\bel} over kalibreringssignalet som er eit relativt svakt signal og det var difor svært underleg om eg skulle få så lågt trykk i vatnet.} og eg såg meg nøydd til å gå igjennom utrekningar og signalkjeda utallige gonger før eg endeleg fann ut at programmet Logic Pro som eg har nytta til opptaka på PC automatisk normaliserte alle opptaka eg hadde gjort. Når eg då fekk slått av denne automatikken, fekk eg straks lydtrykksnsivå over \unit{220}{\deci\bel} noko som stemte overeins med det eg hadde observert på oscilloskopet.