\begin{abstract}
I denne oppgåva har eg sett på korleis ein kan setje opp eit godt måleoppsett for å undersøkje alle relevante parametrar i samband med å fjerne biofilm frå protesar lagt i vatn ved hjelp av ultralyd. I tidlegare undersøkingar har ein i hovudsak nytta ein konstant frekvens og berre variert temperatur og lengda ultralyden har stått på, noko som er svært mangelfullt for å forstå korleis ultralyd påverkar biofilm. Eg har sett på positive og negative sider ved måleoppsett i andre forsøk og prøvd ut eit nytt måleoppsett. Det viste seg at fordelane med tidlegare oppsett var transducerar som kunne stå på over lengre periodar og ein effekt stor nok til å få vatnet til å kavitere. I mitt måleoppsett kunne eg kontrollere frekvens og lydtrykksnivå, men hadde ikkje nok effekt til å få kavitering i vatnet. Ein bør ha eit måleoppsett som tek med seg dei positive kvalitetane til begge måleoppsetta når ein skal gjere vidare forsøk med biofilm, dette fordi ein enno ikkje veit sikkert om det er kavitering eller svingingane som oppstår i samband med ultralyden som fjernar biofilmen. Prosjektet har gjort klar programmert kode til å ta hand om resultat ein får i eit framtidig, betre måleoppsett.
\end{abstract}