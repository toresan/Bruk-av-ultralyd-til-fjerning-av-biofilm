\section{Konklusjon og vidare arbeid}
\subsection{Konklusjon}
Korleis måler så mitt måleoppsett seg opp mot dei ynskjer og krav ein må ha til undersøking av ultralyd sin effekt på biofilm? Måleoppsettet eg har jobba med denne hausten har heilt klart ein del forbetringar det må igjennom før faktiske forsøk med biofilm kan utførast. I hovudsak trur eg det treng fleire transduserar, fleire hydrofonar og ein kraftigare effektforsterkar for å kunne få kraftige nok signal til kavitasjon. Noko anna ein burde forsøke er å nytte transduserar med mykje større bandbreidde enn den eg har nytta. Frekvensresponsen til den nytta transduseren er svært sentrert rundt \unit{38}{\kilo\hertz}, men eg skulle gjerne nytta ein transduser med mykje breiare frekvensrespons for å kunne prøve ut mange ulike frekvensar på biofilmen. 

Måleoppsettet er altså ikkje godt nok slik det er i dag, men med meir og betre utstyr ser eg at det er godt mogleg å få i stand eit godt nok måleoppsett og det er heller ikkje så mykje som skal til for å få det godt nok. I tillegg er all kode i Matlab klar til å ta seg av målingar gjort i eit framtidig måleoppsett noko som er ein fordel.

\subsection{Vidare arbeid}
Framtidig arbeid vil gå ut på å ferdigstille måleoppsettet med kraftigare forsterkar, fleire hydrofonar og transduserar og kanskje ein også kan prøve ein mindre måletank sidan det er ganske mykje tomrom i den eg har nytta til no. Når måleoppsettet så blir som ynskt, vil neste steg vere å gjere mange kontrollerte målingar med biofilm på protesar i desse glaskolbane. Ein bør då ha eit rikt utval i forskjellige bakteriar og ein god prosedyre for å variere alle dei ulike parametrane som kan ha forskjellig effekt på biofilmen. Dei parametrane som då bør bli med er temperatur i vatnet, frekvens, intensitet og varigheit av ultralyden. 