\section{Konklusjon og vidare arbeid}
\subsection{Konklusjon}
Korleis måler så mitt måleoppsett seg opp mot dei ynskjer og krav ein må ha til undersøking av ultralyd sin effekt på biofilm? Måleoppsettet eg har jobba med denne hausten har heilt klart ein del forbetringar det må igjennom før faktiske forsøk med biofilm kan utførast. I hovudsak trur eg det treng fleire transducerar, fleire hydrofonar og ein kraftigare effektforsterkar for å kunne få kraftige nok signal til kavitasjon. Noko anna ein burde forsøke er å nytte transducerar med mykje større bandbreidde enn den eg har nytta. Frekvensresponsen til Skipper $11\text{ x }19^\circ$ er svært sentrert rundt \unit{38}{\kilo\hertz}, men eg skulle gjerne nytta ein transducer med mykje breiare frekvensrespons for å kunne prøve ut mange ulike frekvensar på biofilmen.
Vi kan seie at..

Det ser ut til at..
XXX MEIR HER
Dette fører til at vi må ha (/ har det vi treng)..
 
\subsection{Vidare arbeid}
XXX MÅ OPPDATERE UTSTYRET

XXX SÅ GJERE FAKTISKE MÅLINGAR PÅ BIOFILM
