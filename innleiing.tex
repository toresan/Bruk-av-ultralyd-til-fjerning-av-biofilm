\section{Innleiing}
Dette prosjektet tek føre seg problem og moglegheiter ved eit måleoppsett for å måle alle parametrar som er nødvendig for å kunne finne ut om biofilm på protesar oppbevart i eit reagensglas eller ein glaskolbe kan drepast/fjernast ved hjelp av ultralyd. Ein har allereie hatt noko suksess med å drepe bakteriar ved hjelp av ei maskin som opphavleg er laga for å reinse myntar, men maskina nytta i desse forsøka har ikkje vore optimalisert på nokon måte når det gjeld frekvens, lydtrykk eller temperatur på vatnet \cite{ultraprotese}. Ved å skape ein kontrollert målesituasjon kan vi sjå nærare på dei ulike parametrane som påverkar fjerning av biofilm. Prosjektet skal sjå på kva som har blitt gjort i tidlegare arbeid med ultralyd og biofilm og finne ut kva dette arbeidet har resultert i, då særleg med tanke på instrumentering og forsøksparametrar.

Så kvifor nytte ultralyd til å fjerne biofilm på protesar? Kvifor ikkje desinfiserande middel eller medikament til dette formålet? Det viser seg at det er både kostbart og upraktisk med tidlegare metodar då spesielt i samband med protesar. Årleg er det berre i USA meir enn 650 000 pasientar som får protesar operert inn i kroppen sin, men det er diverre ikkje utan komplikasjonar\cite{ultraprotese}. Protesar som det utviklar seg biofilm på fører til infeksjonar som er smertefullt for pasienten\cite{infection}. Dersom ultralyd viser seg å vere eigna til å fjerne biofilm frå desse protesane, så vil det vere både ein billeg og enkel metode noko som er ynskjeleg for alle.

Vi veit altså allereie at vi kan fjerne enkelte typar bakteriar med ultralyd, men vi veit ikkje kva slags akustiske eigenskapar ved ultralyden det er som er med på å fjerne dei\cite{ultraprotese}. Vi veit ikkje sikkert enno om det er vibrasjonane som oppstår på grunn av det akustiske feltet som losnar bakteriane eller om det er kavitasjon som får biofilmen til å losne. Vi veit heller ikkje kva frekvens som er mest eigna til dette og heller ikkje kva lydtrykk ein må ha for å fjerne biofilmen. Difor er det viktig å ha eit måleoppsett som kan sjå nærare på desse parametrane slik at vi i neste omgang kan gjere forsøk med biofilm for å auke forståinga av effekten ultralyd har på biofilm.