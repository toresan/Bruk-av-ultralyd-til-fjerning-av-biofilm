\section{Resultat}
\subsection{Undersøking av tidlegare arbeid}
I tidlegare forsøk med biofilm og ultralyd har det mellom anna blitt nytta eit ultralydapparat som vanlegvis er nytta innan industrien til å reinse myntar eller klokker \cite{sonorex}. Apparatet i bruk har med andre ord ikkje blitt utvikla spesielt med tanke på proteser og biofilm. Tidlegare forsøk er også stort sett utført av personar med berre medisinsk bakgrunn og ikkje akustisk og forsøka ber preg av manglande innsikt i korleis ultralyd verkar og kva effekt ein kan få av kavitasjon.

\begin{figure}[htbp]
\centering
\includegraphics[width=0.8\textwidth]{bilete/effspektSonorex.pdf}
\caption[Effektspektrum av Sonorexapparat]{Figuren viser effektspektrumet av Sonorexapparatet. Vi ser ein tydeleg topp ved \unit{39}{\kilo\hertz} og ein topp ved 1. harmoniske av denne frekvensen ved \unit{78}{\kilo\hertz}}
\label{fig:effektsonorex}
\end{figure}

Apparatet som er brukt i \cite{ultraprotese} består av ein liten kum til å ha vatn i, 2 transducerar og innebygd varming av vatnet med termostat. Dersom vi ser nærare på apparatet, så ser vi på figur \ref{fig:effektsonorex} at det berre har to innstillingar ein sjølv kan endre på, temperatur og ein timer som bestemmer kor lenge det skal stå på. Ein kan ikkje kontrollere frekvensen eller lydtrykket i tanken. I spesifikasjonane \cite{sonorex} står det at apparatet opererer på \unit{35}{\kilo\hertz}, men målingar med oscilloskop viser at transducerane i tanken opererer på omlag \unit{39}{\kilo\hertz} og at det er tydelige harmoniske frekvensar over dette. Det maksimale lydtrykket i tanken blei målt til \unit{225}{\kilo\pascal} eller \unit{227}{\deci\bel}. Noko som er nok til at XXX vatnet kan kavitere. 

\begin{figure}[htbp]
\centering
\includegraphics[width=0.8\textwidth]{bilete/trykkSonorex.pdf}
\caption[Lydtrykksnivå i Sonorexapparat]{Figuren viser lydtrykksnivået i Sonorexapparatet ved ein topp.}
\label{fig:trykksonorex}
\end{figure}

I desse forsøka blei det sett nærare på korleis temperaturen og lengda i tid biofilmen vart utsett for ultralyd påverka biofilmen. Dette blei gjort for seks ulike bakteriar der det viste seg at nokon var motstandsdyktige uansett tilhøve medan andre andre bakteriar blei drepne under rette tilhøve. Men med berre ein frekvens og eit relativt konstant lydtrykk, så er det uvisst om dette er dei beste tilhøva for å fjerne biofilm frå proteser. Det kan vere at det er betre med andre frekvensar for dei bakteriane som tolte påkjenninga frå ultralyden i denne tanken eller at lydtrykket rett og slett er for lågt. Difor er det nødvendig med ei grundigare undersøking av dei akustiske parametrane.

Det som vil vere viktig å få oversikt over er korleis biofilmen oppfører seg ved ulike frekvensar og korleis varierande lydtrykk kan påverke biofilmen. Samstundes er det viktig å ta med seg dei parametrane som ein nytta i første forsøk. Temperaturen er viktig både med tanke på bakteriane i biofilmen og for kavitasjonsterskelen i vatnet, og ein bør også ta med seg tidsaspektet vidare.

\subsection{Arbeid med eige måleoppsett}
\begin{figure}[htbp]
	\begin{center}
		\subfigure[Sonorexapparat]{
		\label{fig:tidssignal-a}\includegraphics[width=0.3\textwidth]{bilete/tidSonorex.pdf}}
		\subfigure[Hydrofon i vatnet]{
		\label{fig:tidssignal-b}\includegraphics[width=0.3\textwidth]{bilete/tidVatn.pdf}}
		\subfigure[Hydrofon i kolbe i vatnet]{
		\label{fig:tidssignal-c}\includegraphics[width=0.3\textwidth]{bilete/tidGlas.pdf}}
	\end{center}
	\caption[Tidssignala til dei ulike målingane]{Figuren viser utdrag av tidssignala til dei tre ulike målesituasjonane. I min eigen måletank har eg nytta eit burst signal frå signalgeneratoren slik at vi ikkje får eit kontinuerleg signal på same måten som i Sonorexapparatet.}
	\label{fig:tidssignal}
\end{figure}

\begin{figure}[htbp]
\centering
\subfigure[Lydtrykksnivå i måletank utan glaskolbe]{
\label{fig:trykkvatn}
\includegraphics[width=0.45\textwidth]{bilete/trykkVatn.pdf}}
\subfigure[Lydtrykksnivå i måletank med glaskolbe]{
\label{fig:trykkglas}
\includegraphics[width=0.45\textwidth]{bilete/trykkGlas.pdf}}
\caption[Lydtrykksnivå i måletanken]{Her ser vi lydtrykksnivået i måletanken re \unit{1}{\micro\pascal} både når hydrofonen er plassert i ein glaskolbe og når den ikkje er det. Signalgeneratoren sender ut eit burst signal på \unit{38}{\kilo\hertz} med \emph{burst cycle} lik 13.}
\label{fig:trykk}
\end{figure}

\begin{figure}[htbp]
\centering
\includegraphics[width=0.8\textwidth]{bilete/effspektGlas.pdf}
\caption[Effektspektrum av måletank utan glaskolbe]{XXX}
\label{fig:effektvatn}
\end{figure}

\begin{figure}[htbp]
\centering
\includegraphics[width=0.8\textwidth]{bilete/effspektGlas.pdf}
\caption[Effektspektrum av måletank med glaskolbe]{XXX}
\label{fig:effektglas}
\end{figure}



XXX FIGUR: GRAFAR AV RESULTAT

XXX OBSERVERTE IKKJE KAVITASJON

XXX LÅGARE RYKK I KOLBE/REAGENSRØR
