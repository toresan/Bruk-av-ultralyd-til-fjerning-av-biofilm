\section{Resultat}
\subsection{Undersøking av tidlegare arbeid}
I tidlegare forsøk med biofilm og ultralyd har det mellom anna blitt nytta eit ultralydapparat som vanlegvis er nytta innan industrien til å reinse myntar eller klokker \cite{sonorex}. Apparatet i bruk har med andre ord ikkje blitt utvikla spesielt med tanke på proteser og biofilm. Tidlegare forsøk er også stort sett utført av personar med berre medisinsk bakgrunn og ikkje akustisk og forsøka ber preg av manglande innsikt i korleis ultralyd verkar og kva effekt ein kan få av kavitasjon.

\begin{figure}[htbp]
	\centering
	\includegraphics[width=0.8\textwidth]{bilete/effspektSonorex.pdf}
	\caption[Effektspektrum av Sonorexapparat]{Figuren viser effektspektrumet av Sonorexapparatet. Vi ser ein tydeleg topp ved \unit{39}{\kilo\hertz} og ein 	topp ved 1. harmoniske av denne frekvensen ved \unit{78}{\kilo\hertz}. Effekten er normalisert slik at summen av effekten for signalet er lik \unit{0}{\deci		\bel} og aksen viser verdiar i \deci\bel.}
	\label{fig:effektsonorex}
\end{figure}

Apparatet som er brukt i \cite{ultraprotese} består av ein liten kum til å ha vatn i, 2 transducerar og innebygd varming av vatnet med termostat. Dersom vi ser nærare på apparatet, så ser vi at det berre har to innstillingar ein sjølv kan endre på, nemlig temperaturen i vatnet og ein timer som bestemmer kor lenge det skal stå på. Ein kan ikkje kontrollere frekvensen eller lydtrykket i tanken. I spesifikasjonane \cite{sonorex} står det at apparatet opererer på \unit{35}{\kilo\hertz}, men målingar med oscilloskop viser at transducerane i tanken opererer på omlag \unit{39}{\kilo\hertz} og at det er tydelige harmoniske frekvensar over dette. Dette kan ein også sjå på effektspektrumet i figur \ref{fig:effektsonorex}. Det maksimale lydtrykket i tanken blei målt til \unit{225}{\kilo\pascal} eller \unit{227}{\deci\bel}. Ser vi på figur \ref{fig:kavitasjonsterskel}, så ser vi at dette skal vere nok til at vatnet kan kavitere og ein kunne observere små bobler i tanken.

\begin{figure}[htbp]
	\centering
	\includegraphics[width=0.8\textwidth]{bilete/trykkSonorexLang.pdf}
	\caption[Lydtrykksnivå i Sonorexapparat]{Figuren viser lydtrykksnivået i Sonorexapparatet for \unit{50}{\milli\second} og ein ser tydeleg ein stor variasjon i 	lydtrykksnivået.}
	\label{fig:trykksonorexlang}
\end{figure}

I forsøka i \cite{ultraprotese} blei det sett nærare på korleis temperaturen og lengda i tid biofilmen vart utsett for ultralyd påverka biofilmen. Dette blei gjort for seks ulike bakteriartypar der det viste seg at nokon var motstandsdyktige uansett tilhøve medan andre bakteriar blei drepne under rette tilhøve. I Sonorextanken blir ikkje bakeriane utsett for eit konstant lydtrykk, men eit lydtrykk som varierer omlag mellom \unit{205 - 227}{\deci\bel} med ein frekvens på omlag \unit{100}{\hertz}. Dette gjev ein trykkskilnad på heile $10^{\frac{227}{20}-6} - 10^{\frac{205}{20}-6} = \unit{206,1}{\kilo\pascal}$ som er ganske betydeleg. Med berre \emph{ein} frekvens og eit lydtrykk ein ikkje har kontroll over, så er det uvisst om dette er dei beste tilhøva for å fjerne biofilm frå proteser. Det kan vere at det er betre med andre frekvensar for dei bakteriane som tolte påkjenninga frå ultralyden i denne tanken eller at lydtrykket rett og slett er for lågt. Difor er det nødvendig med ei grundigare undersøking av dei akustiske parametrane.

\begin{figure}[htbp]
	\centering
	\includegraphics[width=0.8\textwidth]{bilete/trykkSonorex.pdf}
	\caption[Lydtrykksnivå i Sonorexapparat - lite utdrag]{Figuren viser lydtrykksnivået i Sonorexapparatet ved ein topp.}
	\label{fig:trykksonorex}
\end{figure}

Det som vil vere viktig å få oversikt over er korleis biofilmen oppfører seg ved ulike frekvensar og korleis varierande lydtrykk kan påverke biofilmen. Samstundes er det viktig å ta med seg dei parametrane som ein nytta i første forsøk. Temperaturen er viktig både med tanke på bakteriane i biofilmen og for kavitasjonsterskelen i vatnet, og ein bør også ta med seg tidsaspektet vidare.

\subsection{Arbeid med eige måleoppsett}
I måletanken eg sjølv har sett opp fekk eg ganske andre resultat enn dei i Sonorextanken. Kanskje ikkje så rart sidan det er heilt anna oppsett av transducerar og det at det var avgrensa kor lenge eg kunne sende ut signal til transduceren om gongen. Signalet frå signalgeneratoren blei sendt ut som korte pulsar, såkalla \emph{burst}, som fungerar slik at sinusspenninga berre er på i nokre få perioder av gangen. Det blei nytta eit signal på \unit{38}{\kilo\hertz} med amplitude \unit{1}{\volt}. Som vi ser av figur \ref{fig:tidssignal}, så gir dette ein stor skilnad mellom signalet målt i Sonorextanken og signalet målt i min måletank.

\begin{figure}[htbp]
	\begin{center}
		\subfigure[Sonorexapparat]{
		\label{fig:tidssignal-a}\includegraphics[width=0.3\textwidth]{bilete/tidSonorex.pdf}}
		\subfigure[Hydrofon i vatnet]{
		\label{fig:tidssignal-b}\includegraphics[width=0.3\textwidth]{bilete/tidVatn.pdf}}
		\subfigure[Hydrofon i kolbe i vatnet]{
		\label{fig:tidssignal-c}\includegraphics[width=0.3\textwidth]{bilete/tidGlas.pdf}}
	\end{center}
	\caption[Tidssignala til dei ulike målingane]{Figuren viser utdrag av tidssignala til dei tre ulike målesituasjonane. I min eigen måletank har eg nytta eit burst 	signal frå signalgeneratoren slik at vi ikkje får eit kontinuerleg signal på same måten som i Sonorexapparatet.}
	\label{fig:tidssignal}
\end{figure}

Ein anna stor skilnad er det observerte trykket i min tank og i Sonorextanken. Både med oscilloskop og med verdiar målt til PC er det ein skilnad på omlag \unit{10}{\deci\bel} som tilsvarar ein faktor på 3,16. Det største observerte trykket på oscilloskopet var \unit{89,6}{\kilo\pascal} som i følgje \eqref{kavitasjonsterskel} skal vere nok til at vatnet kan kavitere, men under samtlege målingar observerte eg aldri synlege teikn på at vatnet kaviterte. Eg studerte óg nøye etter teikn på ståande bølgjer i måletanken, men såg ikkje noko som kunne tyde på at det fann stad.

\begin{figure}[htbp]
	\centering
	\subfigure[Lydtrykksnivå i måletank utan glaskolbe]{
		\label{fig:trykkvatn}
		\includegraphics[width=0.45\textwidth]{bilete/trykkVatn.pdf}}
	\subfigure[Lydtrykksnivå i måletank med glaskolbe]{
		\label{fig:trykkglas}
		\includegraphics[width=0.45\textwidth]{bilete/trykkGlas.pdf}}
	\caption[Lydtrykksnivå i måletanken]{Her ser vi lydtrykksnivået i måletanken re \unit{1}{\micro\pascal} både når hydrofonen er plassert i ein glaskolbe og 	når den ikkje er det. Signalgeneratoren sender ut eit burst signal på \unit{38}{\kilo\hertz} med \emph{burst cycle} lik 13 som då varar i omlag \unit{0,34}{\milli	\second} og tidsvindauget vi ser byrjar samstundes med dette burst-signalet.}
	\label{fig:trykk}
\end{figure}

Sidan det er vanleg praksis å leggje proteser på ei eller anna form for vassløysing i ein glaskolbe når ein opererer dei ut i samband med diagnosering av infeksjonar \cite{ultraprotese}, så gjorde eg målingar både med og utan glaskolbe rundt hydrofonen. Glaskolben var fylt med vatn og det blei sørga for at einaste måten lyden kunne gå frå vatnet til hydrofonen var gjennom veggen på glaskolben. Målingane i figur \ref{fig:trykk} viser at lydtrykket inne i glaskolben er noko lågare enn det er utan kolben, noko ein kunne tenkje seg på grunn av transmisjonstap gjennom glaset. Noko ein kunne vere noko mindre sikker på var korleis frekvensspekteret var med og utan glaskolbe, men som figur \ref{fig:effekt} viser så er spekteret ganske likt i begge situasjonane.

\begin{figure}[htbp]
	\centering
	\subfigure[Effektspektrum i måletank utan glaskolbe]{
		\includegraphics[width=0.45\textwidth]{bilete/effspektGlas.pdf}
		\label{fig:effektvatn}}
	\subfigure[Effektspektrum i måletank med glaskolbe]{
		\includegraphics[width=0.45\textwidth]{bilete/effspektGlas.pdf}
		\label{fig:effektglas}}
	\caption[Effektspektrum av hydrofon i måletank]{Figuren viser oss effektspektrum frå hydrofonen med og utan glaskolbe. Det er ein tydeleg topp ved \unit	{38}{\kilo\hertz} som venta med ein del støy i resten av frekvensspekteret.}
	\label{fig:effekt}
\end{figure}